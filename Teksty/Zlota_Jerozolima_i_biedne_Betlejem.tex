\beginsong{Złota Jerozolima i biedne Betlejem}[by={Jacek Zielinski, Leszek A. Moczulski}]
\beginverse
\chordson
\[a]Zawieja \[e]i bezna\[a]dzieja,
\[a]Złota Je\[D]rozo\[E]lima,
A w \[F]biednym \[C]Betlejem \[d]Pa\[a]ni
Sy\[F]neczka w \[e]grocie po\[a]wiła.
\chordsoff
\vspace{0.2cm}
Zima się ludzi trzyma,
Złote denary wszędzie,
Bieda straszna w Betlejem,
Tam tylko Pan przybędzie!
\endverse
\beginchorus
\chordson
\[a]Do szopy, \[e]do szopy, \[a]wszyscy,
Kto \[C]ogrzać \[D]pragnie \[E]ręce,
Z da\[F]rami, z da\[G]rami, z da\[a]rami,
By \[F]odta\[G]jało \[a]serce!
\chordsoff
\endchorus
\beginverse
Nie do złota i mirry,
Nie do różanych pachnideł,
Ale do szopki w zawieję
Tej nocy Pan do nas przybył.
\vspace{0.2cm}
Nie do wieży z księgami,
Gdzie mędrcy świata najwięksi,
Ale w szopie, w zawieję,
Gdzie pastuszkowie są pierwsi.
\endverse
\endsong