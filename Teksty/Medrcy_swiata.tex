\beginsong{Mędrcy świata, monarchowie}[by={XVII wiek}]
\beginverse
\chordson
\[A]Mędrcy świata, \[D]monar\[A]chowie,
\[h]gdzie śpiesz\[E]nie dą\[A]życie?
\[A]Powiedzcież nam \[D]Trzej Królowie,
\[A]chcecie \[E]widzieć \[E7]Dzie\[A]cię?
\[E7]Ono w żłobie \[f#]nie ma \[A]tronu
\[E7(D)]i ber\[A]ła nie \[E]dzierży.
\[A]a proroctwo \[D]Jego zgonu,
\[A]już się w \[E7]świecie sze\[A]rzy.
\chordsoff
\endverse
\beginverse
Mędrcy świata złość okrutna,
Dziecię prześladuje,
wieść okropna, wieść to smutna,
Herod spisek knuje.
Nic monarchów nie odstrasza,
do Betlejem śpieszą,
gwiazda Zbawcę im ogłasza,
nadzieją się cieszą.
\endverse
\beginverse
Przed Maryją stają społem,
niosą Panu dary,
przed Jezusem biją czołem,
składają ofiary.
Trzykroć szczęśliwi Królowie,
któż wam nie zazdrości?
Cóż my damy, kto nam powie,
pałając z miłości.
\endverse
\endsong