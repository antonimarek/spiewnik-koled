\beginsong{Mizerna cicha, stajenka licha (a)}[by={sł. Teofil Lenartowicz, muz. Jan Gall}]
\beginverse
Mizerna, cicha stajenka licha,
pełna niebieskiej chwały.
Oto leżący przed nami śpiący,
w promieniach Jezus mały.
\endverse
\beginverse
Przed nim anieli w locie stanęli
i pochyleni klęczą.
Z włosy złotymi, skrzydły białymi,
pod malowaną tęczą.
\endverse
\beginverse
Wielkie zdziwienie, wszelkie stworzenie,
cały świat orzeźwiony.
Mądrość mądrości, światłość światłości,
Bóg człowiek tu wcielony.
\endverse
\beginverse
I oto mnodzy, ludzie ubodzy,
radzi oglądać Pana.
Pełni natchnienia, pewni zbawienia,
upadli na kolana.
\endverse
\beginverse
Długo czekali, długo wzdychali,
Aż niebo rozgorzało,
Piekło zawarte, niebo otwarte,
Słowo się Ciałem stało.
\endverse
\beginverse
Hej ludzie prości Bóg z nami gości,
skończony czas niedoli.
On daje Siebie chwała na niebie,
pokój ludziom dobrej woli.
\endverse
\endsong