\beginsong{Mędrcy świata, monarchowie}[by={XVII wiek}]
\beginverse
Mędrcy świata, monarchowie,
gdzie śpiesznie dążycie?
Powiedzcież nam Trzej Królowie,
chcecie widzieć Dziecię?
Ono w żłobie nie ma tronu
i berła nie dzierży.
a proroctwo Jego zgonu,
już się w świecie szerzy.
\endverse
\beginverse
Mędrcy świata złość okrutna,
Dziecię prześladuje,
wieść okropna, wieść to smutna,
Herod spisek knuje.
Nic monarchów nie odstrasza,
do Betlejem śpieszą,
gwiazda Zbawcę im ogłasza,
nadzieją się cieszą.
\endverse
\beginverse
Przed Maryją stają społem,
niosą Panu dary,
przed Jezusem biją czołem,
składają ofiary.
Trzykroć szczęśliwi Królowie,
któż wam nie zazdrości?
Cóż my damy, kto nam powie,
pałając z miłości.
\endverse
\endsong