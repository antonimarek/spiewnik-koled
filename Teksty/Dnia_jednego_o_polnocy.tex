\beginsong{Dnia jednego o północy}
\beginverse
Dnia jednego o północy,
gdym zasnął w wielkiej niemocy;
nie wiem, czy na jawie, czy mi się śniło,
że wedle mej budy słońce świeciło.
\endverse
\beginverse
Sam się czemprędzej porwałem
i na drugich zawołałem:
na Kubę, na Maćka i na Kaźmierza,
by wstali czem prędzej mówić pacierza.
\endverse
\beginverse
Nie zaraz się podźwignęli,
bo byli twardo zasnęli:
alem ich po trochu wziął za czuprynę,
by wstali przywitać Boga dziecinę.
\endverse
\beginverse
Kaźmierz bowiem wszystko słyszał,
Bo na słomie w budzie dyszał:
Ale nam od strachu nie chciał powiedzieć,
Na Maćka skazował, ten musi wiedzieć.
\endverse
\beginverse
Mój Macieju, ty nam powiesz,
Ponieważ ty sam wszystko wiesz:
Widziałem, widziałem dziwne widzenie,
Słyszałem, słyszałem Anielskie pienie.
\endverse
\beginverse
Bo mi sam Anioł powiedział,
Gdym na słomie w budzie siedział:
Nie bój się, nie bój się Maćku pastuszka,
I jać to, i jać to jest boski służka.
\endverse
\beginverse
Zwiastujęć wesołe lata,
Że się wam Zbawiciel świata
W Betleem narodził, tak sławnem mieście,
Więc jego czemprędzej przywitać bieżcie.
\endverse
\beginverse
Niech weźmie Stasiek fujarę,
A Szymek gołąbków parę,
A Maciek będzie stał u drzwi z obuchem,
Bo się tam nie zmieści z swym wielkim brzuchem.
\endverse
\beginverse
Porwawszy się biegli drogą,
gdzie widzieli jasność srogą:
w Betleem miasteczku gdzie dziecię było,
które się dla wszystkich z nieba zjawiło.
\endverse
\beginverse
Wbiegliśmy zaraz do szopy,
uściskaliśmy Mu stopy;
jam dobył fujary, a Kuba rogu,
graliśmy wesoło na chwałę Bogu.
\endverse
\endsong